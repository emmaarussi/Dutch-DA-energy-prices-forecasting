\documentclass{article}
\usepackage{amsmath}
\usepackage{listings}
\usepackage{xcolor}
\usepackage{hyperref}
\usepackage{graphicx}

\definecolor{codegreen}{rgb}{0,0.6,0}
\definecolor{codegray}{rgb}{0.5,0.5,0.5}
\definecolor{codepurple}{rgb}{0.58,0,0.82}
\definecolor{backcolour}{rgb}{0.95,0.95,0.92}

\lstdefinestyle{mystyle}{
    backgroundcolor=\color{backcolour},   
    commentstyle=\color{codegreen},
    keywordstyle=\color{magenta},
    numberstyle=\tiny\color{codegray},
    stringstyle=\color{codepurple},
    basicstyle=\ttfamily\footnotesize,
    breakatwhitespace=false,         
    breaklines=true,                 
    captionpos=b,                    
    keepspaces=true,                 
    numbers=left,                    
    numbersep=5pt,                  
    showspaces=false,                
    showstringspaces=false,
    showtabs=false,                  
    tabsize=2
}

\lstset{style=mystyle}

\title{Documentation: Recursive Autoregressive Model for Energy Price Forecasting}
\author{Dutch Energy Price Analysis}
\date{\today}

\begin{document}

\maketitle

\section{Model Overview}
The simple_arp_recursive.py implements a recursive autoregressive model for forecasting Dutch energy prices. This model focuses on medium to long-term forecasting horizons (14-38 hours ahead) using only historical price values as input features.

\section{Model Architecture}

\subsection{Autoregressive Framework}
The model uses the statsmodels AutoReg class to implement an autoregressive model of order p, where p is automatically selected based on the data. The model can be represented as:

\begin{equation}
y_t = c + \sum_{i=1}^{p} \phi_i y_{t-i} + \epsilon_t
\end{equation}

where:
\begin{itemize}
    \item $y_t$ is the price at time t
    \item $c$ is the intercept term
    \item $\phi_i$ are the autoregressive coefficients
    \item $p$ is the order of the model (automatically selected)
    \item $\epsilon_t$ is the error term
\end{itemize}

\section{Training Process}

\subsection{Data Preparation}
\begin{itemize}
    \item Training period: January 2023 to December 31, 2023
    \item Test period: January 1, 2024 to February 28, 2024
    \item Data frequency: Hourly
    \item Price variable: price_eur_per_mwh
\end{itemize}

\subsection{Lag Selection}
The model implements automatic lag selection through the following process:
\begin{enumerate}
    \item Initially tests lags from 1 to 71 hours
    \item Uses AIC (Akaike Information Criterion) to select optimal lags
    \item If automatic selection fails, defaults to lag=24 (daily seasonality)
\end{enumerate}

\section{Forecasting Process}

\subsection{Daily Forecasting Routine}
\begin{enumerate}
    \item For each day at 12:00 in the test period:
    \begin{itemize}
        \item Model is fitted on all available data up to the forecast start time
        \item Predictions are made for t+14h, t+24h, and t+38h horizons
        \item Predictions are made recursively, using previous predictions as inputs
    \end{itemize}
    \item Each forecast uses the following steps:
    \begin{enumerate}
        \item Initialize with the last known values
        \item For each step up to the target horizon:
        \begin{itemize}
            \item Calculate next prediction using AR coefficients
            \item Update history with the prediction
            \item Continue until reaching target horizon
        \end{itemize}
    \end{enumerate}
\end{enumerate}

\subsection{Recursive Prediction}
The recursive prediction process can be represented as:

\begin{equation}
\hat{y}_{t+h} = c + \sum_{i=1}^{p} \phi_i y_{t+h-i}
\end{equation}

where:
\begin{itemize}
    \item For $h \leq 0$: $y_{t+h}$ are actual historical values
    \item For $h > 0$: $y_{t+h}$ are predicted values
\end{itemize}

\section{Model Evaluation}

\subsection{Performance Metrics}
For each forecast horizon (14h, 24h, 38h), the following metrics are calculated:
\begin{itemize}
    \item RMSE (Root Mean Square Error)
    \item SMAPE (Symmetric Mean Absolute Percentage Error)
    \item R² (Coefficient of Determination)
    \item Mean actual vs. predicted prices
\end{itemize}

\subsection{Stationarity Analysis}
The model performs stationarity tests on both training and test data:
\begin{itemize}
    \item Augmented Dickey-Fuller (ADF) test
    \item Autocorrelation Function (ACF) plots up to 100 lags
\end{itemize}

\section{Visualization}
The model generates the following visualizations:
\begin{itemize}
    \item Actual vs. predicted prices for each horizon
    \item ACF plots for training and test periods
    \item All plots are saved in the models_14_38/ar/plots_arp_recursive/ directory
\end{itemize}

\section{Implementation Details}

\subsection{Key Functions}
\begin{itemize}
    \item \textbf{forecast\_day():} Main forecasting function that implements the AR model
    \item \textbf{main():} Orchestrates the entire forecasting process
\end{itemize}

\subsection{Dependencies}
\begin{itemize}
    \item statsmodels.tsa.ar\_model.AutoReg
    \item statsmodels.tsa.stattools.adfuller
    \item statsmodels.graphics.tsaplots.plot\_acf
    \item pandas, numpy, matplotlib
\end{itemize}

\section{Usage}
To run the model:
\begin{lstlisting}[language=bash]
python models_14_38/ar/simple_arp_recursive.py
\end{lstlisting}

The script will:
\begin{enumerate}
    \item Load the preprocessed multivariate features
    \item Train the model and make predictions
    \item Generate performance metrics and plots
    \item Save results in the specified directories
\end{enumerate}

\end{document}
